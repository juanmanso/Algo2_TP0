\documentclass[11pt, spanish]{report}
\usepackage[spanish]{babel}
\selectlanguage{spanish}
\usepackage[utf8]{inputenc}
\usepackage{graphicx}
\usepackage{subcaption}
\usepackage{amsmath}
\usepackage[margin=2.5cm]{geometry}
\usepackage{afterpage}
%\usepackage[spanish,onelanguage]{algorithm2e}
\usepackage{algpseudocode}

\renewcommand{\algorithmicwhile}{\textbf{mientras}}
\renewcommand{\algorithmicdo}{\textbf{hacer}}
\renewcommand{\algorithmicfor}{\textbf{para}}
\renewcommand{\algorithmicreturn}{\textbf{devolver}}
\renewcommand{\algorithmicend}{\textbf{fin}}

\newcommand{\rpm}{\raisebox{.2ex}{$\scriptstyle\pm$}}

%-------------------------------------------------------
% Aca empiezan las deifiniciones para listings con color

\usepackage{listings}
\usepackage{color}

\definecolor{mygreen}{rgb}{0,0.6,0}
\definecolor{mygray}{rgb}{0.5,0.5,0.5}
\definecolor{mymauve}{rgb}{0.58,0,0.82}

\lstset{ %
  backgroundcolor=\color{white},   % choose the background color; you must add \usepackage{color} or \usepackage{xcolor}
  basicstyle=\footnotesize,        % the size of the fonts that are used for the code
  breakatwhitespace=false,         % sets if automatic breaks should only happen at whitespace
  breaklines=true,                 % sets automatic line breaking
  captionpos=b,                    % sets the caption-position to bottom
  commentstyle=\color{mygreen},    % comment style
  deletekeywords={...},            % if you want to delete keywords from the given language
  escapeinside={\%*}{*)},          % if you want to add LaTeX within your code
  extendedchars=true,              % lets you use non-ASCII characters; for 8-bits encodings only, does not work with UTF-8
  frame=single,	                   % adds a frame around the code
  keepspaces=true,                 % keeps spaces in text, useful for keeping indentation of code (possibly needs columns=flexible)
  keywordstyle=\color{blue},       % keyword style
  language=C++,                 % the language of the code
  otherkeywords={*,...},           % if you want to add more keywords to the set
  numbers=left,                    % where to put the line-numbers; possible values are (none, left, right)
  numbersep=5pt,                   % how far the line-numbers are from the code
  numberstyle=\tiny\color{mygray}, % the style that is used for the line-numbers
  rulecolor=\color{black},         % if not set, the frame-color may be changed on line-breaks within not-black text (e.g. comments (green here))
  showspaces=false,                % show spaces everywhere adding particular underscores; it overrides 'showstringspaces'
  showstringspaces=false,          % underline spaces within strings only
  showtabs=false,                  % show tabs within strings adding particular underscores
  stepnumber=1,                    % the step between two line-numbers. If it's 1, each line will be numbered
  stringstyle=\color{mymauve},     % string literal style
  tabsize=4,	                   % sets default tabsize to 2 spaces
  title=\lstname                   % show the filename of files included with \lstinputlisting; also try caption instead of title
}

\begin{document}


\title{{\Huge Trabajo Práctico 1} \\[2pt] {\huge Procesamiento de señales AM.}}
\author{{\Large Juan Manso, 96133}  \\ {\Large Ricardo Andreasen, 96322}}
\date{}
\maketitle

{\huge Introducción}

\paragraph{} En el presente informe se detalla la implementación de una herramienta para procesar señales moduladas en amplitud, de acuerdo al siguiente esquema:

\begin{figure}[h]
\centering
\includegraphics[width = 0.6 \textwidth]{flujo_senales}
\caption{Flujo de procesamiento de señales.}
\label{fig:flujo_senales}
\end{figure}

Se provee como entrada una secuencia de números complejos $(I[n],Q[n])$ que representan las muestras en fase y cuadratura de la señal modulada. Esta secuencia se pasa por un \textit{moving average}, un decimador y finalmente se le calcula el módulo punto a punto para obtener a la salida una secuencia de números reales $R[n']$.
 
\paragraph{}

{\huge Algoritmo}

\paragraph{} En primer lugar se definió el algoritmo mediante el cual se realizaría el procesamiento de señales. 

\paragraph{} Se tuvo en cuenta el hecho de que, a partir del \textit{downsampling}, se desecha una cantidad importante de muestras. Específicamente, si se fijó un parámetro de decimación $N$, de cada $N$ muestras procesadas sólo una es finalmente utilizada. Por lo tanto sólo hace falta realizar los cálculos para las muestras que sabemos van a ``sobrevivir'' la decimación.


\paragraph{} Para cada muestra $n$, el \textit{moving average} realiza un promedio de las $N$ últimas muestras recibidas, es decir sigue la ecuación:

$$ y[n] = \frac1N \sum_{k = 0}^{N-1} x[n-k]. $$

De lo anterior se sigue que cada muestra a la salida $y[n_0]$ es sólo función de los $N$ últimos valores a la entrada $x[n_0], x[n_0 - 1], \dots,  x[n_0 - N + 1]$. Los demás bloques de procesamiento sólo dependen del valor actual de su entrada, por lo que finalmente en el programa sólo hace falta recordar las últimas $N$ muestras que se recibieron para realizar los cálculos. 

\paragraph{} Con estas observaciones en mente, se optó por el siguiente algoritmo de implementación:

\begin{algorithmic}[H] % algorithmic fue el que pude hacer andar, porque me dejaba 
		       % redefinir las keywords. algorithm2e era mas lindo pero seteandolo
                       % con spanish y onelanguage mezclaba español con italiano.
		       % Si encontras uno que se vea mejor y en español es bienvenido.

%\caption{Algoritmo para el procesamiento de las muestras.}

\While{el \textit{stream} de datos no termine}
	
	\State $promedio := 0$
	
	\For{las N siguientes muestras}

		\State $promedio \gets promedio + x[n]$
	\EndFor
	
	\State $promedio \gets promedio / N$

\Return $\sqrt{\mathbf{Re}\{promedio\}^2 + \mathbf{Im}\{promedio\}^2}$ 
		
\EndWhile

\end{algorithmic}

\newpage

{\huge Implementación}

\paragraph{} De acuerdo a lo requerido, la implementación de la herramienta se realizó en C++, para entradas y salidas de texto en los formatos especificados. Para ello se aprovecharon las clases \texttt{cmdline} y \texttt{complex} provistas en el curso, para el manejo de argumentos y números complejos respectivamente.

\paragraph{} Con respecto al manejo de argumentos, para utilizar la clase \texttt{cmdline} hace falta definir la tabla de opciones que se esperan recibir, de tipo \texttt{option\_t}. Se optó por definirla dentro del \texttt{main} del programa, con las siguientes opciones:

\lstset{language=C++}
\begin{lstlisting}[frame=single]
static option_t options[] = {
	{1, "i", "input", "-", opt_input, OPT_DEFAULT},
	{1, "o", "output", "-", opt_output, OPT_DEFAULT},
	{1, "N", "n_decimator","500",opt_n_decimator, OPT_DEFAULT},
	{0, "h", "help", NULL, opt_help, OPT_DEFAULT},
	{0, },
\end{lstlisting}

Como es requerido, se definió a la cadena ``-'' (tercera columna de la tabla de opciones) como el valor por defecto para las opciones de entrada y de salida. De esta manera, ya sea que se haya omitido esa opción o se la especifique explicítamente para que asuma su valor por defecto, en ambos casos se  obtendrá el mismo resultado.

Además la clase \texttt{cmdline} pide especificar las funciones que se utilizarán para parsear cada opción (5ta columna de la tabla) - se las definió en el mismo \texttt{main}. Para las funciones que parsean las opciones de entrada, salida y ayuda se aprovecharon las que ya estaban implementadas en el archivo \texttt{main.cc} provisto con la implementación de la clase \texttt{cmdline} del curso. Quedó por definir, entonces, el \texttt{parser} para el parámetro $N$ de decimación:


\lstset{language=C++} % Esta seteado para utf8 pero no puede renderear los acentos! No entiendo por que
\begin{lstlisting}[frame=single]
static void
opt_n_decimator(string const &arg)
{
	istringstream iss(arg);

	// Intentamos extraer el N de la línea de comandos.
	// Para detectar argumentos que únicamente consistan de
	// números enteros, vamos a verificar que EOF llegue justo
	// después de la lectura exitosa del escalar.
	
	if(!(iss >> n_decimator) || !iss.eof()) {
		cerr << "non-integer factor: "
		     << arg
		     << "."
		     << endl;
		exit(1);
	}

	if (iss.bad()) {
		cerr << "cannot read integer factor."
		     << endl;
		exit(1);
	}
}
\end{lstlisting}

Como el método \texttt{parse} de \texttt{cmdline} ya le pasa al \texttt{parser} su valor por defecto si fue omitido, siempre esta función recibirá un argumento para $N$ (no necesariamente válido). Se intenta leer el mismo como un entero, y de fallar se anuncia el error y se termina el programa. 

Los argumentos se leen mediante \texttt{cmdline.parse($\cdot$)} a cada una de las variables estáticas definidas fuera del \texttt{main}:

 
\lstset{language=C++}
\begin{lstlisting}[frame=single]
static size_t n_decimator;	// Decimator Factor (factor positivo de decimación)
static istream *iss = 0;	// Input Stream (clase para manejo de los flujos de entrada)
static ostream *oss = 0;	// Output Stream (clase para manejo de los flujos de salida)
static fstream ifs; 		// Input File Stream (derivada de la clase ifstream que deriva de istream para el manejo de archivos)
static fstream ofs;		// Output File Stream (derivada de la clase ofstream que deriva de ostream para el manejo de archivos)
\end{lstlisting}

Estas son las variables que se le pasan a \texttt{am\_proc} que es la encargada de implementar el algoritmo de la sección anterior, de manera que el \texttt{main} queda:


\lstset{language=C++}
\begin{lstlisting}[frame=single]
int
main(int argc, char * const argv[])
{
	cmdline cmdl(options);	// Objeto con parametro tipo option_t (struct) declarado globalmente.
	cmdl.parse(argc, argv);	// Metodo de parseo de la clase cmdline
	am_proc(iss, oss, n_decimator);	// Procesamiento AM
}
\end{lstlisting}

La función \texttt{am\_proc} está declarada en ``\texttt{am\_proc.h}'' y definida en ``\texttt{am\_proc.cc}''. Fuera de chequear por errores de lectura, su implementación es en escencia el pseudo-código de la sección anterior:

\lstset{language=C++}
\begin{lstlisting}[frame=single]
void am_proc(istream *is, ostream *os, const size_t& n_decimator){
	
	bool eof_flag=false;
	size_t i;
	complejo c, aux; // aux_c tendrá la suma y aux_x será el que recibe el complejo del stream


	// Si entra un archivo vacio (primero lee EOF), corta el for y luego el while, devolviendo un vacio

	while(!eof_flag){
		
		//  Se suman los primeros 'n_decimator' números hasta que corte
		for(i=1; i<=n_decimator && ((*is)>>aux); i++)
			c += aux;
	
		// Compruebo si se llegó a EOF
		if(is->eof())
			eof_flag=true;

		if(is->bad()){ 
		// El for terminó por no puder guardar el caracter en x
			cerr	<< "Error: Cannot read complex on input stream"
				<< endl;
			exit(1);
		}		

		// Realizo el promedio móvil
		c=c / n_decimator;
			
		// Imprimo el valor absoluto
		*os << c.abs()<<endl;
	}
	
	if(os->bad()){
		cerr	<< "Error: Cannot write output file"
			<< endl;
		exit(1);
	}

}
\end{lstlisting}

{\huge Compilación}

\paragraph{}

{\huge Corridas de prueba}

\paragraph{}

{\huge Conclusiones}

\paragraph{}

\end{document}
