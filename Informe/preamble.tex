%Preambulo para articulo científico de LaTeX

\usepackage[a4paper,left=3cm,right=3cm,bottom=3.5cm,top=3.5cm]{geometry} 	% Configuro la geometría del papel
%\usepackage{microtype}														% Mejora el "spacing" de las palabras
\usepackage[spanish]{babel} 												% Compatibilizo los signos del español
	\addto\captionsspanish{\renewcommand{\tablename}{Tabla}}				%% Redefino nombres preestablecidos por Babel
	\addto\captionsspanish{\renewcommand{\listtablename}{Índice de tablas}}	%% y así en vez de Cuadro dirá Tabla.
\usepackage{amsmath, amsfonts, amssymb}										% Entornos matemáticos, fuentes y símbolos
\usepackage{graphicx}														% Necesario para insertar figuras
\usepackage{fancyhdr}														% Para manipular headers y footers
\usepackage[usenames]{color}											% \color{color deseado} {lo que querés que tenga color}
\usepackage{subcaption}														% Permite captions del tipo 1a, 1b
\usepackage{multirow}														% Para tablas
\usepackage{float}

\ifListings
	\usepackage{listingsutf8}

	\definecolor{mygreen}{rgb}{0,0.6,0}
	\definecolor{mygray}{rgb}{0.5,0.5,0.5}
	\definecolor{mymauve}{rgb}{0.58,0,0.82}
	
	\lstset{
		backgroundcolor=\color{white},   % choose the background color; you must add \usepackage{color} or \usepackage{xcolor}
		inputencoding=utf8,
		basicstyle=\footnotesize,        % the size of the fonts that are used for the code
		breakatwhitespace=false,         % sets if automatic breaks should only happen at whitespace
		breaklines=true,                 %% sets automatic line breaking
		captionpos=b,                    %% sets the caption-position to bottom
		commentstyle=\color{mygreen},    % comment style
		deletekeywords={...},            % if you want to delete keywords from the given language
		escapeinside={\%*}{*)},          % if you want to add LaTeX within your code
		extendedchars=true,              % lets you use non-ASCII characters; for 8-bits encodings only, does not work with UTF-8
		frame=single,	                 %% adds a frame around the code
		keepspaces=true,                 % keeps spaces in text, useful for keeping indentation of code (possibly needs columns=flexible)
		keywordstyle=\color{blue},       % keyword style
		language=C++,		 	 %% the language of the code
		otherkeywords={*,...},           % if you want to add more keywords to the set
		numbers=left,                    %% where to put the line-numbers; possible values are (none, left, right)
		numbersep=5pt,                   %% how far the line-numbers are from the code
		numberstyle=\tiny\color{mygray}, % the style that is used for the line-numbers
		rulecolor=\color{black},         % if not set, the frame-color may be changed on line-breaks within not-black text (e.g. comments (green here))
		showspaces=false,                % show spaces everywhere adding particular underscores; it overrides 'showstringspaces'
		showstringspaces=false,          % underline spaces within strings only
		showtabs=false,                  % show tabs within strings adding particular underscores
		stepnumber=1,                    % the step between two line-numbers. If it's 1, each line will be numbered
		stringstyle=\color{mymauve},     % string literal style
		tabsize=4,	                   % sets default tabsize to 2 space
		title=\lstname                   %% show the filename of files included with \lstinputlisting; also try caption instead of title
	}
	 \usepackage{algpseudocode}						% Para pseudocodigo
	 \renewcommand{\algorithmicwhile}{\textbf{mientras}} 
	 \renewcommand{\algorithmicdo}{\textbf{hacer}} 
	 \renewcommand{\algorithmicfor}{\textbf{para}}
	 \renewcommand{\algorithmicreturn}{\textbf{devolver}}
	 \renewcommand{\algorithmicend}{\textbf{fin}} 
	 
	 \newcommand{\rpm}{\raisebox{.2ex}{$\scriptstyle\pm$}}  
\fi

\ifSiunix
\usepackage{siunitx}											% Unidades: \SI {cantidad} {\unidad} (necesita texlive-science)
	\sisetup{load-configurations = abbreviations}							% Habilita poner \cm en vez de \centi\metre
	\sisetup{output-decimal-marker = {,}}									% Cambia los puntos decimales por comas
\fi

\usepackage{booktabs}														% Permite hacer tablas sin separadores en el medio
\usepackage{placeins}														
		\let\Oldsection\section												%% Permite que los flotantes (como figuras) no aparescan
	\renewcommand{\section}{\FloatBarrier\Oldsection}						%% antes o después de su sección correspondiente.
		\let\Oldsubsection\subsection
	\renewcommand{\subsection}{\FloatBarrier\Oldsubsection}		
		\let\Oldsubsubsection\subsubsection
	\renewcommand{\subsubsection}{\FloatBarrier\Oldsubsubsection}
\usepackage{hyperref}														% Debe ser agregado al final del preambulo

\hypersetup
{    bookmarks=true,         % show bookmarks bar?
     unicode=false,          % non-Latin characters in Acrobat’s bookmarks
     pdftoolbar=true,        % show Acrobat’s toolbar?
     pdfmenubar=true,        % show Acrobat’s menu?
     pdffitwindow=false,     % window fit to page when opened
     pdftitle={\myTitle},    		 % title
     pdfauthor={\myAuthorSurname},   % author
	 pdfcreator={\myAuthorSurname},	 % creator = author
     pdfsubject={\mySubject},		 % subject of the document
     pdfkeywords={\myKeywords},
     colorlinks=true,        % false: boxed links; true: colored links
     linkcolor=black,        % color of internal links (change box color with linkbordercolor)
     citecolor=black,        % color of links to bibliography
     filecolor=magenta,      % color of file links
     urlcolor=cyan           % color of external links
}

%Configuro la pagina con los encabezaos y pies de paginas
\pagestyle{fancy}										% Para agregar encabezados y pie de paginas	
\lhead{\mySubject}										% Encabezado izquierdo
\rhead{\includegraphics[scale=0.15]{\myHeaderLogo}} 	% Encabezado derecho (logo de la FIUBA)					
